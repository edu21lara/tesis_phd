\documentclass[upright, contnum]{umemoria}

%fix for the oneside argument
\makeatletter
\g@addto@macro\titlepage{\pagenumbering{Alph}}
\g@addto@macro\endtitlepage{\pagenumbering{roman}}
\makeatother

\depto{Departamento de Ingeniería Industrial}
\author{Andrea Ignacia Canales Gutierrez}
\title{Essays on Anti-Competitive Behavior}
\auspicio{}
\date{Octubre 2018}
\guia{Juan Escobar Castro}
\carrera{Doctora en Sistemas de Ingeniería}
\memoria{Tesis para optar al Grado de \break  Doctora en Sistemas de Ingeniería}
\comision{Leonardo Basso Sotz, Gastón Llanes, Matteo Triossi Verondini}

\usepackage{lipsum}

\usepackage[utf8]{inputenc}
\usepackage[T1]{fontenc}

\begin{document}

\frontmatter
\maketitle

\begin{abstract}
{En esta tesis, a raíz de dos problemas distintos, se estudia cómo la entrada de nuevas firmas a un mercado puede modificar los incentivos a desarrollar conductas anticompetitivas por parte de la o las firmas incumbentes. En primer lugar, estudiamos un problema de colusión con entrada de nuevas firmas. Por otra parte, se modelan los incentivos a utilizar estrategias predatorias cuando los costos de producción pueden disminuir o aumentar, dependiendo si logran acumular experiencia en el mercado o no.
En el capítulo 1 se estudian los carteles incompletos. La evidencia muestra que, en muchos casos, los acuerdos colusivos no se conforman con la participación de todas las firmas del mercado si no que, un grupo de firmas convive con una franja competitiva de empresas, estos se denominan carteles incompletos. En este contexto estudiamos el acuerdo colusivo óptimo, donde las variaciones de precio, por parte de las firmas que lo fijan, pueden generar entrada o salida de firmas de la franja, y por tanto, cambiar la estructura del mercado. En este modelo, una firma que se desvía del acuerdo colusivo puede tener incentivos a fijar un precio muy bajo, con el objetivo de sacar del mercado firmas de la franja competitiva. Esta salida se produce solo si las firmas de la franja esperan un castigo severo por parte del cartel ante el desvío, es decir,una etapa de castigo donde se observarían precios muy bajos. Este trabajo identifica un nuevo trade-off entre la severidad del castigo y las ganancias del desvío de un acuerdo colusivo.
En el capítulo 2 se aborda un problema de predación. Se ha documentado que en al- gunos mercados, a medida que las firmas acumulan experiencia en la producción de bienes y/o servicios, disminuyen sus costos, este fenómeno de denomina learning-by-doing. Por otro lado, también se documenta que la pérdida de experiencia produce el fenómeno contrario, llamado organizational forgetting. En este capítulo de desarrolla un modelo general en un mercado donde pueden coexistir ambos efectos y se estudian los incentivos a utilizar estra- tegias predatorias. Se considera un modelo asimétrico, donde la firma incumbente tiene un mercado cautivo, y puede forzar la salida de una firma entrante. Se observa que ambas firmas pueden utilizar estrategias agresivas de precios pero con distintos objetivos: la incumbente para forzar la salida de la entrante, y esta última para evitar ser predada. Además, para un caso reducido, observamos que la interacción de learning-by-doing y organizational forgetting provoca una competencia en precios menos agresiva que si solo está presente uno de estos efectos en el mercado.}
\end{abstract}

\begin{dedicatoria} % opcional
Una dedicatoria corta. Por ejemplo, al \emph{Centro Tecnológico Ucampus}
\end{dedicatoria}

\begin{thanks} % opcional
\lipsum[1-2]
\end{thanks}
\cleardoublepage

\tableofcontents
\listoftables % opcional
\listoffigures % opcional

\mainmatter

\begin{intro}

    In the last decade, the advancement of data science has significantly transformed various sectors, enabling more informed and efficient decision-making. In the public sector, the application of data analysis techniques offers unprecedented opportunities to improve public policy management, optimize resources, and respond more effectively to complex challenges. However, despite its potential, the implementation of data-based approaches in the public sector faces multiple obstacles, including limitations in technological infrastructure, cultural barriers, and the need for policies that ensure transparency and fairness in the use of information.
    
    The COVID-19 pandemic has highlighted the importance of agile and evidence-based decision-making in the public sector. Managing the health crisis has required implementing measures like lockdowns and mobility restrictions, whose effectiveness has varied considerably depending on socioeconomic factors. In Santiago, Chile, the capital of a country with significant socioeconomic inequalities, these measures have shown mixed results. Using detailed geolocated mobile phone data, mobility patterns during lockdowns were analyzed, revealing that high-income areas reduced their mobility by 50–90\% mainly through voluntary confinement behaviors, compared to 20–50\% in low-income communities. These findings were developed collaboratively with other researchers \cite{carranza2022social}, establishing a direct relationship between increased mobility and higher infection rates, where a 10\% increase in mobility is associated with a 5\% increase in infection rates. This work not only highlighted the importance of socioeconomic disparities in compliance with health measures but also provided insights that informed the implementation of more targeted public health policies.

    Furthermore, improving public procurement processes has been key to reducing costs and making government spending more efficient. In Chile, the framework agreements managed by ChileCompra represent a considerable portion of public spending, specifically 23\% of procurement expenses during 2018-19. However, inefficiencies were identified related to low competition in the auction processes of some framework agreements, which could potentially increase government costs. Framework agreements (FAs) are procurement mechanisms that offer an intermediate approach between centralized and decentralized purchasing, allowing a central agency to select a range of products and suppliers through auctions. Despite their advantages in terms of cost control and leveraging economies of scale, the lack of competition in auctions can lead to higher prices and increased costs for the government.
    
    To address these challenges, collaboration with ChileCompra was undertaken to redesign the FAs, implementing data-based reforms that included the standardization of product catalogs through natural language processing (NLP) algorithms and promoting greater competition in supplier auctions. The redesign process and its outcomes were developed in partnership with other researchers \cite{olivares2024saving}, demonstrating that the pilot redesign of the Food framework agreement achieved an 8\% reduction in transaction prices. This motivated ChileCompra to implement the developed tools more broadly, estimating a savings of approximately \$74 million in 2022. These findings provided significant guidelines for the practical implementation of FAs and contributed to the literature on applied market design.

    This research not only contributes to the existing state of the art but also has practical implications for those responsible for formulating public policies in Chile and in other similar contexts. By showing how data-based approaches can improve the efficiency and effectiveness of government interventions, this thesis offers a replicable model to face future challenges in the public sector. In the field of public health, the findings on mobility and socioeconomic disparities can inform more equitable and effective strategies to manage future health crises. In the area of public procurement, the proposed reforms are serving as a guide for other institutions seeking to optimize their processes and reduce costs through data-based approaches. Additionally, thanks to this work, modifications were made to the State procurement and contracting law, approved by the Senate of Chile in June 2023, which will allow for a simpler and more direct implementation of the tools developed in this work.
    
    In summary, this thesis explores how data science can transform decision-making in the public sector through two case studies focused on managing the COVID-19 pandemic and optimizing public procurement processes in Chile. The findings highlight the value of data-based approaches to address complex challenges, improving public policy outcomes and generating significant savings for the state. Additionally, the organization of the thesis is detailed: Chapter 1 details the study on optimizing public procurement, focusing on the framework agreements managed by ChileCompra and the reforms implemented to improve competition in auctions and reduce transaction costs. On the other hand, Chapter 2 explores in depth the management of the COVID-19 pandemic, analyzing mobility patterns and their impact on infection rates, as well as socioeconomic disparities in compliance with health measures. Each chapter presents the methodology employed, the results obtained, and the practical implications of the findings, providing a comprehensive view of how data science can improve decision-making in the public sector.
    
    
    
\end{intro}

\input{cap1.tex}
\input{cap2.tex}
\input{conclu.tex}

% \input{glosario.tex} % opcional

\bibliographystyle{plain}
\bibliography{bibliografia}

% \chapter*{Annexes}
\addcontentsline{toc}{chapter}{Annexes}
\chapter*{Further details of methodologies, equations and robustness analysis}


\section{Econometric Model and Data Processing} \label{app:data}

\subsection{Model specification}
This section provides a detailed description of the econometric models that were estimated using a difference-in-difference design.

Let the index $t\in\{\text{Old},\text{New}\}$ represent the old (2014) and the new (2017) Food FAs, respectively.  Let $b_{ijrt}$ represent the bid offered in FA $t$ by supplier $j$ to provide product $i$ in region $r$, including shipping.  Define $B^{med}_{irt}$ as the \textit{median} bid across all supplier bids submitted in the FA for that product-region-FA combination. In calculating this median bid price, we considered alternative methods for discarding extreme prices that could be generated by bidding mistakes or unrealistically aggressive bidding. For the main results we used a modified Tukey rule \citep{Tukey1977}, described in the \textit{Data collection and processing} section. To make the bids for the 2014 and 2017 Food FAs comparable, all prices for 2014 were adjusted using the CPI food price index. In addition, we normalized bid prices so that they represented bid prices per unit. See the \textit{Data collection and processing} section for further details on data pre-processing. 

The following difference-in-differences specification is used to estimate the effect of the competitive treatment condition ---denoted by the indicator variable $Comp_{i}$--- on bid prices:
\begin{equation}
    \log (B^{med}_{irt}) = \delta_r + \gamma_i + \alpha New_{t} + \beta New_{t}\times Comp_{i} + \varepsilon_{irt} \ ,
    \label{eq:reg_sub_bid}
\end{equation}

\noindent where $New_t$ is an indicator equal to one for the new FA and $\delta_r$ is a region fixed effect. The sample is comprised of all products $i$ that were matched across the old and new FAs based on similar attributes, and the regression includes a product fixed effect $\gamma_i$. {The error term $\varepsilon_{irt}$ represents other idiosyncratic unobservable factors that affect bid prices of each product across FAs and  regions.} The coefficient $\alpha$ measures the average differences in bid prices between the old and new FAs for the noncompetitive baseline group. The key parameter of interest is $\beta$, the coefficient capturing the incremental change in bid prices for the auctions in the competitive treatment condition in the new FA.

The second model specification uses panel data on posted and transaction prices. Let $p_{ijrtw}$ denote the average posted prices by supplier $j$ for product $i$ in region $r$ during calendar week $w$ in FA $t$. As with the bids, we calculated the {\em median posted price} for a product across all suppliers during that week, denoted by $P^{med}_{irtw}$. That is, $P^{med}_{irtw}$ denotes the median price at which product $i$ could be purchased in region $r$ during calendar week $w$ under the operation of FA $t$ in the period listed above. 

We estimated the panel regression:
\begin{equation}
    \log (P^{med}_{irtw})= \delta_r + \gamma_i + \tau_{wc(i)} + \alpha New_{t} + \beta New_{t}\times Comp_{i} + \varepsilon_{irtw} \ ,
    \label{eq:reg_op_posted}
\end{equation}
\noindent where $\delta_r$ and $\gamma_i$ are region and product fixed effects respectively, $\tau_{wc(i)}$ is a product category-specific calendar week dummy variable to capture potential seasonality ($c(i)$ indicates the category of product $i$). {The error term $\varepsilon_{irtw}$ represents other unobservable factors that affect prices during the operation stage.} All the models were estimated using Ordinary Least Squares and robust standard errors. 

\subsection{Data collection and processing}
In order to compare prices between different FA and to rule out unreasonable values, we establish a procedure to process bids, transaction prices, and posted prices.

\paragraph{Bids.}  In the case of bids, we identify outliers within each distribution of bids defined by a standardized product $i$ and FA $t$. In the case of FA 2014, we use prices before adding the respective shipping rate. In the case of FA 2017, we normalize prices to the Metropolitan Region (where the capital city Santiago is located) for outlier identification. Additionally, we use the CPI index to adjust prices from FA 2014 for inflation. In Chile there are available CPI indexes that are product specific, which we use to adjust the prices offered in 2014 to present value (to the year 2017), and therefore each product category is adjusted differently according to the corresponding specific CPI. 

We identify four types of outliers, using four different approaches. For each FA $t$ and standardized product $i$, we define a set of similar products $C_{i,t}$. (A standardized product was defined in the main text.) First, we compute the mean price $\hat{p}_{i,t}$ of its distribution. For each supplier $s$, product $j$, region $r$, and FA $t$, with bid $p_{s,j,r,t}$, we compute the ratio $R_{s,j,r,t} = \frac{p_{s,j,r,t}}{\hat{p}_{i,t}}$, where $i$ is the standardized product associated with $j$. We define outliers as follows:

\begin{itemize}
\item[] \textbf{Type 1}: $p_{s,j,r,t}$ is an outlier if $R_{s,j,r,t}<1/2$ or $R_{s,j,r,t}>2$.
\item[] \textbf{Type 2}: $p_{s,j,r,t}$ is an outlier if $R_{s,j,r,t}<1/3$ or $R_{s,j,r,t}>3$.
\item[] \textbf{Type 3}: $p_{s,j,r,t}$ is identified as an outlier by Tukey's rule: observations that more than 1.5$\times$IQR (inter-quartile range) below and above the first and third quartile, respectively.
\item[] \textbf{Type 4}: If 20\% of the products in the set of similar products $C_{i,t}$ meet the Type 3 outlier criterion, all products in the set are considered outliers. That is, we exclude all the standardized products that exhibit a large proportion (20\% or more) of extreme values.%\todoMO{Re-wrote explanation}
\end{itemize}

First, to compute normalized bids or prices, we divide prices by the relevant number of units in the bundle.
In addition, as a robustness check we estimate regression \eqref{eq:reg_sub_bid} by correcting prices by volume to take into account potential scale effects such as volume discounts. See Tables \ref{tab:estsub_vol} and \ref{tab:estadj_vol}. Thus, prices of products specifying larger quantities of items would be comparable to those of products with a single or a few items. Specifically, in FA 2014 products were sold in large bundles. To account for potential volume discounts, we estimate 
 the regression: 
\begin{equation}
    \log (B^{med}_{irt}) = \delta_r + \gamma_i + \beta \log(units_{irt}) + \varepsilon_{irt} \ ,
    \label{eq:reg_sub_bid_rob}
\end{equation}
where $units_{irt}$ represents the amount of items per bundle for product $i$. Then, we compute the adjusted bid price  $\hat{B}^{med}_{irt} = \log B^{med}_{irt} - \hat{\beta}\log(units_{irt})$ and use this as the dependent variable in regression \eqref{eq:reg_sub_bid}.

\paragraph{Transaction and posted prices.} Regarding transaction and posted prices, we correct by inflation according to the monthly CPI index for Food and Non-alcoholic Beverages, using as the reference month December 2013. For transaction and posted prices, we identify three types of outliers. Let be $C(i,t)$ the set of transaction/posted prices of a standardized product $i$ in FA $t$. For each transaction/post $k$ we define $R_{k,t} = \frac{p_{k,t}}{\hat{p}_{i,t}}$, where $p_{k,t}$ is the unit price of $k$, and $\hat{p}_{i,t}$ is the average price in $C_{i,t}$. Then, we define outliers as follows:

\begin{itemize}
\item[] \textbf{Type 1}: If the proportion of transactions/posts with $R_{k,t}<1/2$ or $R_{k,t}>2$ is greater than 20\%.
\item[] \textbf{Type 2}: If the proportion of transactions/posts with $R_{k,t}<1/3$ or $R_{k,t}>3$ is greater than 20\%.
\item[] \textbf{Type 3}: If the proportion of outliers in $C_{i,t}$ is greater than 20\% by Tukey's rule.
\end{itemize}

As a robustness check, in the \textit{Robustness analysis and alternative regression specifications} section we show the results under the different outlier elimination methods.


\section{Robustness analysis and alternative regression 
 specifications}\label{app:robustness}

%- Results with different forms of outliers
%- Results with TP clustering
%- Results with log(q) adjustments for bids.

\subsection{Regressions for submitted bids}



\begin{table}[H]
   \centering
   \begin{tabular}{lcccc}
      \toprule
                                & Type 1         & Type 2         & Type 3         & Type 4 \\   
                                & (1)            & (2)            & (3)            & (4)\\  
      \midrule 
      New                       & -0.148$^{***}$ & -0.151$^{***}$ & -0.146$^{***}$ & -0.141$^{***}$\\   
                                & (0.005)        & (0.005)        & (0.005)        & (0.006)\\   
      New $\times$ Comp  & -0.029$^{***}$ & -0.034$^{***}$ & 0.009          & 0.002\\   
                                & (0.007)        & (0.008)        & (0.009)        & (0.010)\\   
       \\
      Observations              & 13,111         & 13,195         & 13,163         & 12,349\\  
      R$^2$                     & 0.94644        & 0.93650        & 0.92588        & 0.92289\\  
      Adjusted R$^2$            & 0.94433        & 0.93401        & 0.92297        & 0.91965\\  
      \bottomrule
   \end{tabular}
   
   \par \raggedright 
   \caption{Estimation for submitted bids under different outlier elimination rules. This table shows estimates for submitted bids according to  \eqref{eq:reg_sub_bid}. Each model is defined by the type of method used for outlier detection according to section \ref{app:data}. Standardized product and region fixed effect are included in all Models. We compute \textit{robust} standard errors.}
\end{table}




\begin{table}[H]
   \label{app:tab:volume_adj}
   \bigskip
   \centering
   \begin{tabular}{lcccc}
      \toprule
                                & Type 1         & Type 2         & Type 3         & Type 4 \\   
                                & (1)            & (2)            & (3)            & (4)\\  
      \midrule 
      New                       & -0.125$^{***}$ & -0.128$^{***}$ & -0.176$^{***}$ & -0.177$^{***}$\\   
                                & (0.005)        & (0.005)        & (0.005)        & (0.005)\\   
      New $\times$ Comp  & -0.023$^{***}$ & -0.028$^{***}$ & 0.001          & -0.008\\   
                                & (0.007)        & (0.008)        & (0.009)        & (0.009)\\   
       \\
      Observations             & 13,111         & 13,195         & 13,163         & 12,349\\  
      R$^2$                     & 0.94616        & 0.93629        & 0.92623        & 0.92344\\  
      Adjusted R$^2$            & 0.94403        & 0.93379        & 0.92333        & 0.92023\\  
      \bottomrule
   \end{tabular}
   
   \par \raggedright 
   \caption{\label{tab:estsub_vol} Estimation for submitted bids: Adjusting by volume. This table shows estimates for submitted bids  when prices are adjusted by volume following the specification in \eqref{eq:reg_sub_bid_rob}. Each model is defined by the type of method used for outlier detection according defined in Appendix~\ref{app:data}. Standardized product and region fixed effect are included in all models. We compute robust standard errors.}
\end{table}





\begin{table}[H]
   \centering
   \begin{tabular}{lcccc}
      \toprule
       & Type 1         & Type 2         & Type 3         & Type 4 \\ & (1)            & (2)            & (3)            & (4)\\  
      \midrule 
      New                       & -0.123$^{***}$ & -0.125$^{***}$ & -0.118$^{***}$ & -0.115$^{***}$\\   
                                & (0.005)        & (0.005)        & (0.006)        & (0.006)\\   
      New $\times$ Comp  & -0.041$^{***}$ & -0.048$^{***}$ & -0.001         & -0.012\\   
                                & (0.008)        & (0.009)        & (0.010)        & (0.010)\\   
       \\
      Observations              & 11,760         & 11,933         & 11,901         & 11,382\\  
      R$^2$                     & 0.94768        & 0.93700        & 0.92620        & 0.92320\\  
      Adjusted R$^2$            & 0.94537        & 0.93426        & 0.92297        & 0.91969\\  
      \bottomrule
   \end{tabular}
   \par \raggedright 
   \caption{Estimation for submitted bids under different outlier elimination rules. This table shows estimates for submitted bids according to  regression \eqref{eq:reg_sub_bid} for auctions that awarded at least one bid. Each model is defined by the type of method used for outlier detection according to section \ref{app:data}. Standardized product and region fixed effect are included in all Models. We compute \textit{robust} standard errors.}
\end{table}




\subsection{Regressions for awarded bids}


\begin{table}[H]
   \centering
   \begin{tabular}{lcccc}
      \toprule
                                & Type 1         & Type 2         & Type 3         & Type 4 \\   
                                & (1)            & (2)            & (3)            & (4)\\  
      \midrule 
      New                       & -0.072$^{***}$ & -0.073$^{***}$ & -0.064$^{***}$ & -0.055$^{***}$\\   
                                & (0.005)        & (0.005)        & (0.006)        & (0.007)\\   
      New $\times$ Comp  & -0.162$^{***}$ & -0.173$^{***}$ & -0.074$^{***}$ & -0.081$^{***}$\\   
                                & (0.008)        & (0.009)        & (0.012)        & (0.013)\\   
       \\
      Observations              & 11,760         & 11,933         & 11,901         & 11,382\\  
      R$^2$                     & 0.94555        & 0.93189        & 0.89614        & 0.89202\\  
      Adjusted R$^2$            & 0.94314        & 0.92893        & 0.89160        & 0.88708\\  
      \bottomrule
   \end{tabular}
   
   \par \raggedright 
   \caption{Estimation for awarded bids  under different outlier elimination rules. This table shows estimates for awarded bids. Each model is defined by the type of method used for outlier detection according to Appendix \ref{app:data}. Standardized product and region fixed effect are included in all models. We compute \textit{robust} standard errors.}
\end{table}




\begin{table}[H]
   \label{app:tab:volume_adj_awarded}
   \centering
   \begin{tabular}{lcccc}
      \toprule
                                & Type 1         & Type 2         & Type 3         & Type 4 \\   
                                & (1)            & (2)            & (3)            & (4)\\  
      \midrule 
      New                       & -0.033$^{***}$ & -0.035$^{***}$ & -0.105$^{***}$ & -0.105$^{***}$\\   
                                & (0.005)        & (0.005)        & (0.006)        & (0.007)\\   
      New $\times$ Comp  & -0.156$^{***}$ & -0.167$^{***}$ & -0.082$^{***}$ & -0.092$^{***}$\\   
                                & (0.008)        & (0.009)        & (0.011)        & (0.012)\\   
       \\
      Observations              & 11,760         & 11,933         & 11,901         & 11,382\\  
      R$^2$                     & 0.94506        & 0.93146        & 0.89717        & 0.89362\\  
      Adjusted R$^2$            & 0.94263        & 0.92848        & 0.89268        & 0.88875\\  
      \bottomrule
   \end{tabular}
   
   \par \raggedright 
   \caption{\label{tab:estadj_vol} Estimation for awarded bids: Adjusting by volume. This table shows estimates for awarded bids when prices are adjusted by volume following \eqref{eq:reg_sub_bid_rob}. Each model is defined by the type of method used for outlier detection according to Appendix~\ref{app:data}. Standardized product and region fixed effect are included in all models. We compute \textit{robust} standard errors.}
\end{table}




\subsection{Regressions for posted prices}


\begin{table}[H]
   \centering
   \begin{tabular}{lccc}
      \toprule
                                & Type 1         & Type 2         & Type 3 \\   
                                & (1)            & (2)            & (3)\\  
      \midrule 
      New                       & -0.029$^{***}$ & -0.033$^{***}$ & -0.025$^{***}$\\   
                                & (0.0007)       & (0.0008)       & (0.0008)\\   
      New $\times$ Comp  & -0.076$^{***}$ & -0.075$^{***}$ & -0.092$^{***}$\\   
                                & (0.0009)       & (0.0009)       & (0.001)\\   
       \\
      Observations              & 1,012,598      & 1,018,778      & 973,195\\  
      R$^2$                     & 0.97025        & 0.96865        & 0.96132\\  
      Adjusted R$^2$            & 0.97018        & 0.96857        & 0.96121\\  
      \bottomrule
   \end{tabular}
   
   \par \raggedright 
   \caption{Estimation for posted prices  under different outlier elimination rules. This table shows estimates for posted prices as in \eqref{eq:reg_op_posted}. Each Model is defined by the type of method used for outlier detection according to Appendix~\ref{app:data}. Standardized product, region, and week category-specific fixed effect are included in all models. We compute \textit{robust} standard errors.}
\end{table}



\subsection{Regressions for transaction prices}


\begin{table}[H]
   \centering
   \begin{tabular}{lccc}
      \toprule
                                & Type 1         & Type 2         & Type 3 \\   
                                & (1)            & (2)            & (3)\\  
      \midrule 
      New                       & -0.007$^{***}$ & -0.006$^{***}$ & 0.004$^{**}$\\   
                                & (0.002)        & (0.002)        & (0.002)\\   
      New $\times$ Comp  & -0.068$^{***}$ & -0.071$^{***}$ & -0.082$^{***}$\\   
                                & (0.002)        & (0.002)        & (0.002)\\   
       \\
      Observations              & 193,702        & 194,312        & 180,421\\  
      R$^2$                     & 0.97441        & 0.97328        & 0.97395\\  
      Adjusted R$^2$            & 0.97409        & 0.97294        & 0.97360\\  
      \bottomrule
   \end{tabular}
   
   \par \raggedright 
   \caption{Estimation for transaction prices. This table shows estimates for transaction prices as defined in Equation \eqref{eq:reg_op_posted}. Each model is defined by the type of method used for outlier detection according to Appendix~\ref{app:data}. Standardized product, region and week category-specific fixed effect are included in all models. We compute \textit{robust} standard errors.}
\end{table}



 % opcionales

\end{document}
