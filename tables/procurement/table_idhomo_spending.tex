\begin{table}[H]
\centering 
\renewcommand{\TPTminimum}{\linewidth}
\small{
\begin{threeparttable}
\makebox[\linewidth]{%
%\setlength\extrarowheight{-3pt}
\begin{tabular}{@{\extracolsep{-5pt}} p{8cm}ll} 
\\\hline 
 & \multicolumn{2}{c}{No. of Products (\% sales)}\\ 
 \cline{2-3}
 & FA 2014 & FA 2017 \\ 
\hline \\[-2.5ex] 
Total number of \textit{pantry} products & 8,937 (100\%) & 3,789 (100\%) \\
Products with identified attributes & 7,642 (92.4\%) & 3,704 (97.3\%) \\
Unique products based on selected attributes & 2,703 (92.4\%) & 2,027 (97.3\%) \\
Matched products across FAs & 928 (68.2\%) & 928 (74.1\%) \\ 
\hline 
\end{tabular}}
% \begin{tablenotes}
%   \small
%      \item \textit{Note:} This table summarizes the results of the product standardization process. We say that a SKU is feasible if all the following attributes were identified: \textit{subcategory}, \textit{type}, \textit{brand} and \textit{format}. The mentioned attributes define a unique Standardized product.
%    \end{tablenotes}
\end{threeparttable}
}
\caption{Product identification summary for all \textit{pantry} products {-- packaged products which can be described through standardized attributes --} in the 2014 (old) and 2017 (new) FAs. The last row counts unique products that were found in both FAs.}
\label{tab:matching_results}
\end{table} 
