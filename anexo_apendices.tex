\newpage

\appendix

\chapter{Further details of methodologies, equations and robustness analysis}

\section{Econometric Model and Data Processing} \label{app:data}

\subsection{Model specification}
This section provides a detailed description of the econometric models that were estimated using a difference-in-difference design.

Let the index $t\in\{\text{Old},\text{New}\}$ represent the old (2014) and the new (2017) Food FAs, respectively.  Let $b_{ijrt}$ represent the bid offered in FA $t$ by supplier $j$ to provide product $i$ in region $r$, including shipping.  Define $B^{med}_{irt}$ as the \textit{median} bid across all supplier bids submitted in the FA for that product-region-FA combination. In calculating this median bid price, we considered alternative methods for discarding extreme prices that could be generated by bidding mistakes or unrealistically aggressive bidding. For the main results we used a modified Tukey rule \citep{Tukey1977}, described in the \textit{Data collection and processing} section. To make the bids for the 2014 and 2017 Food FAs comparable, all prices for 2014 were adjusted using the CPI food price index. In addition, we normalized bid prices so that they represented bid prices per unit. See the \textit{Data collection and processing} section for further details on data pre-processing. 

The following difference-in-differences specification is used to estimate the effect of the competitive treatment condition ---denoted by the indicator variable $Comp_{i}$--- on bid prices:
\begin{equation}
    \log (B^{med}_{irt}) = \delta_r + \gamma_i + \alpha New_{t} + \beta New_{t}\times Comp_{i} + \varepsilon_{irt} \ ,
    \label{eq:reg_sub_bid}
\end{equation}

\noindent where $New_t$ is an indicator equal to one for the new FA and $\delta_r$ is a region fixed effect. The sample is comprised of all products $i$ that were matched across the old and new FAs based on similar attributes, and the regression includes a product fixed effect $\gamma_i$. {The error term $\varepsilon_{irt}$ represents other idiosyncratic unobservable factors that affect bid prices of each product across FAs and  regions.} The coefficient $\alpha$ measures the average differences in bid prices between the old and new FAs for the noncompetitive baseline group. The key parameter of interest is $\beta$, the coefficient capturing the incremental change in bid prices for the auctions in the competitive treatment condition in the new FA.

The second model specification uses panel data on posted and transaction prices. Let $p_{ijrtw}$ denote the average posted prices by supplier $j$ for product $i$ in region $r$ during calendar week $w$ in FA $t$. As with the bids, we calculated the {\em median posted price} for a product across all suppliers during that week, denoted by $P^{med}_{irtw}$. That is, $P^{med}_{irtw}$ denotes the median price at which product $i$ could be purchased in region $r$ during calendar week $w$ under the operation of FA $t$ in the period listed above. 

We estimated the panel regression:
\begin{equation}
    \log (P^{med}_{irtw})= \delta_r + \gamma_i + \tau_{wc(i)} + \alpha New_{t} + \beta New_{t}\times Comp_{i} + \varepsilon_{irtw} \ ,
    \label{eq:reg_op_posted}
\end{equation}
\noindent where $\delta_r$ and $\gamma_i$ are region and product fixed effects respectively, $\tau_{wc(i)}$ is a product category-specific calendar week dummy variable to capture potential seasonality ($c(i)$ indicates the category of product $i$). {The error term $\varepsilon_{irtw}$ represents other unobservable factors that affect prices during the operation stage.} All the models were estimated using Ordinary Least Squares and robust standard errors. 

\subsection{Data collection and processing}
In order to compare prices between different FA and to rule out unreasonable values, we establish a procedure to process bids, transaction prices, and posted prices.

\paragraph{Bids.}  In the case of bids, we identify outliers within each distribution of bids defined by a standardized product $i$ and FA $t$. In the case of FA 2014, we use prices before adding the respective shipping rate. In the case of FA 2017, we normalize prices to the Metropolitan Region (where the capital city Santiago is located) for outlier identification. Additionally, we use the CPI index to adjust prices from FA 2014 for inflation. In Chile there are available CPI indexes that are product specific, which we use to adjust the prices offered in 2014 to present value (to the year 2017), and therefore each product category is adjusted differently according to the corresponding specific CPI. 

We identify four types of outliers, using four different approaches. For each FA $t$ and standardized product $i$, we define a set of similar products $C_{i,t}$. (A standardized product was defined in the main text.) First, we compute the mean price $\hat{p}_{i,t}$ of its distribution. For each supplier $s$, product $j$, region $r$, and FA $t$, with bid $p_{s,j,r,t}$, we compute the ratio $R_{s,j,r,t} = \frac{p_{s,j,r,t}}{\hat{p}_{i,t}}$, where $i$ is the standardized product associated with $j$. We define outliers as follows:

\begin{itemize}
\item[] \textbf{Type 1}: $p_{s,j,r,t}$ is an outlier if $R_{s,j,r,t}<1/2$ or $R_{s,j,r,t}>2$.
\item[] \textbf{Type 2}: $p_{s,j,r,t}$ is an outlier if $R_{s,j,r,t}<1/3$ or $R_{s,j,r,t}>3$.
\item[] \textbf{Type 3}: $p_{s,j,r,t}$ is identified as an outlier by Tukey's rule: observations that more than 1.5$\times$IQR (inter-quartile range) below and above the first and third quartile, respectively.
\item[] \textbf{Type 4}: If 20\% of the products in the set of similar products $C_{i,t}$ meet the Type 3 outlier criterion, all products in the set are considered outliers. That is, we exclude all the standardized products that exhibit a large proportion (20\% or more) of extreme values.%\todoMO{Re-wrote explanation}
\end{itemize}

First, to compute normalized bids or prices, we divide prices by the relevant number of units in the bundle.
In addition, as a robustness check we estimate regression \eqref{eq:reg_sub_bid} by correcting prices by volume to take into account potential scale effects such as volume discounts. See Tables \ref{tab:estsub_vol} and \ref{tab:estadj_vol}. Thus, prices of products specifying larger quantities of items would be comparable to those of products with a single or a few items. Specifically, in FA 2014 products were sold in large bundles. To account for potential volume discounts, we estimate 
 the regression: 
\begin{equation}
    \log (B^{med}_{irt}) = \delta_r + \gamma_i + \beta \log(units_{irt}) + \varepsilon_{irt} \ ,
    \label{eq:reg_sub_bid_rob}
\end{equation}
where $units_{irt}$ represents the amount of items per bundle for product $i$. Then, we compute the adjusted bid price  $\hat{B}^{med}_{irt} = \log B^{med}_{irt} - \hat{\beta}\log(units_{irt})$ and use this as the dependent variable in regression \eqref{eq:reg_sub_bid}.

\paragraph{Transaction and posted prices.} Regarding transaction and posted prices, we correct by inflation according to the monthly CPI index for Food and Non-alcoholic Beverages, using as the reference month December 2013. For transaction and posted prices, we identify three types of outliers. Let be $C(i,t)$ the set of transaction/posted prices of a standardized product $i$ in FA $t$. For each transaction/post $k$ we define $R_{k,t} = \frac{p_{k,t}}{\hat{p}_{i,t}}$, where $p_{k,t}$ is the unit price of $k$, and $\hat{p}_{i,t}$ is the average price in $C_{i,t}$. Then, we define outliers as follows:

\begin{itemize}
\item[] \textbf{Type 1}: If the proportion of transactions/posts with $R_{k,t}<1/2$ or $R_{k,t}>2$ is greater than 20\%.
\item[] \textbf{Type 2}: If the proportion of transactions/posts with $R_{k,t}<1/3$ or $R_{k,t}>3$ is greater than 20\%.
\item[] \textbf{Type 3}: If the proportion of outliers in $C_{i,t}$ is greater than 20\% by Tukey's rule.
\end{itemize}

As a robustness check, in the \textit{Robustness analysis and alternative regression specifications} section we show the results under the different outlier elimination methods.


\section{Robustness analysis and alternative regression 
 specifications}\label{app:robustness}

%- Results with different forms of outliers
%- Results with TP clustering
%- Results with log(q) adjustments for bids.

\subsection{Regressions for submitted bids}



\begin{table}[H]
   \centering
   \begin{tabular}{lcccc}
      \toprule
                                & Type 1         & Type 2         & Type 3         & Type 4 \\   
                                & (1)            & (2)            & (3)            & (4)\\  
      \midrule 
      New                       & -0.148$^{***}$ & -0.151$^{***}$ & -0.146$^{***}$ & -0.141$^{***}$\\   
                                & (0.005)        & (0.005)        & (0.005)        & (0.006)\\   
      New $\times$ Comp  & -0.029$^{***}$ & -0.034$^{***}$ & 0.009          & 0.002\\   
                                & (0.007)        & (0.008)        & (0.009)        & (0.010)\\   
       \\
      Observations              & 13,111         & 13,195         & 13,163         & 12,349\\  
      R$^2$                     & 0.94644        & 0.93650        & 0.92588        & 0.92289\\  
      Adjusted R$^2$            & 0.94433        & 0.93401        & 0.92297        & 0.91965\\  
      \bottomrule
   \end{tabular}
   
   \par \raggedright 
   \caption{Estimation for submitted bids under different outlier elimination rules. This table shows estimates for submitted bids according to  \eqref{eq:reg_sub_bid}. Each model is defined by the type of method used for outlier detection according to section \ref{app:data}. Standardized product and region fixed effect are included in all Models. We compute \textit{robust} standard errors.}
\end{table}




\begin{table}[H]
   \label{app:tab:volume_adj}
   \bigskip
   \centering
   \begin{tabular}{lcccc}
      \toprule
                                & Type 1         & Type 2         & Type 3         & Type 4 \\   
                                & (1)            & (2)            & (3)            & (4)\\  
      \midrule 
      New                       & -0.125$^{***}$ & -0.128$^{***}$ & -0.176$^{***}$ & -0.177$^{***}$\\   
                                & (0.005)        & (0.005)        & (0.005)        & (0.005)\\   
      New $\times$ Comp  & -0.023$^{***}$ & -0.028$^{***}$ & 0.001          & -0.008\\   
                                & (0.007)        & (0.008)        & (0.009)        & (0.009)\\   
       \\
      Observations             & 13,111         & 13,195         & 13,163         & 12,349\\  
      R$^2$                     & 0.94616        & 0.93629        & 0.92623        & 0.92344\\  
      Adjusted R$^2$            & 0.94403        & 0.93379        & 0.92333        & 0.92023\\  
      \bottomrule
   \end{tabular}
   
   \par \raggedright 
   \caption{\label{tab:estsub_vol} Estimation for submitted bids: Adjusting by volume. This table shows estimates for submitted bids  when prices are adjusted by volume following the specification in \eqref{eq:reg_sub_bid_rob}. Each model is defined by the type of method used for outlier detection according defined in Appendix~\ref{app:data}. Standardized product and region fixed effect are included in all models. We compute robust standard errors.}
\end{table}





\begin{table}[H]
   \centering
   \begin{tabular}{lcccc}
      \toprule
       & Type 1         & Type 2         & Type 3         & Type 4 \\ & (1)            & (2)            & (3)            & (4)\\  
      \midrule 
      New                       & -0.123$^{***}$ & -0.125$^{***}$ & -0.118$^{***}$ & -0.115$^{***}$\\   
                                & (0.005)        & (0.005)        & (0.006)        & (0.006)\\   
      New $\times$ Comp  & -0.041$^{***}$ & -0.048$^{***}$ & -0.001         & -0.012\\   
                                & (0.008)        & (0.009)        & (0.010)        & (0.010)\\   
       \\
      Observations              & 11,760         & 11,933         & 11,901         & 11,382\\  
      R$^2$                     & 0.94768        & 0.93700        & 0.92620        & 0.92320\\  
      Adjusted R$^2$            & 0.94537        & 0.93426        & 0.92297        & 0.91969\\  
      \bottomrule
   \end{tabular}
   \par \raggedright 
   \caption{Estimation for submitted bids under different outlier elimination rules. This table shows estimates for submitted bids according to  regression \eqref{eq:reg_sub_bid} for auctions that awarded at least one bid. Each model is defined by the type of method used for outlier detection according to section \ref{app:data}. Standardized product and region fixed effect are included in all Models. We compute \textit{robust} standard errors.}
\end{table}




\subsection{Regressions for awarded bids}


\begin{table}[H]
   \centering
   \begin{tabular}{lcccc}
      \toprule
                                & Type 1         & Type 2         & Type 3         & Type 4 \\   
                                & (1)            & (2)            & (3)            & (4)\\  
      \midrule 
      New                       & -0.072$^{***}$ & -0.073$^{***}$ & -0.064$^{***}$ & -0.055$^{***}$\\   
                                & (0.005)        & (0.005)        & (0.006)        & (0.007)\\   
      New $\times$ Comp  & -0.162$^{***}$ & -0.173$^{***}$ & -0.074$^{***}$ & -0.081$^{***}$\\   
                                & (0.008)        & (0.009)        & (0.012)        & (0.013)\\   
       \\
      Observations              & 11,760         & 11,933         & 11,901         & 11,382\\  
      R$^2$                     & 0.94555        & 0.93189        & 0.89614        & 0.89202\\  
      Adjusted R$^2$            & 0.94314        & 0.92893        & 0.89160        & 0.88708\\  
      \bottomrule
   \end{tabular}
   
   \par \raggedright 
   \caption{Estimation for awarded bids  under different outlier elimination rules. This table shows estimates for awarded bids. Each model is defined by the type of method used for outlier detection according to Appendix \ref{app:data}. Standardized product and region fixed effect are included in all models. We compute \textit{robust} standard errors.}
\end{table}




\begin{table}[H]
   \label{app:tab:volume_adj_awarded}
   \centering
   \begin{tabular}{lcccc}
      \toprule
                                & Type 1         & Type 2         & Type 3         & Type 4 \\   
                                & (1)            & (2)            & (3)            & (4)\\  
      \midrule 
      New                       & -0.033$^{***}$ & -0.035$^{***}$ & -0.105$^{***}$ & -0.105$^{***}$\\   
                                & (0.005)        & (0.005)        & (0.006)        & (0.007)\\   
      New $\times$ Comp  & -0.156$^{***}$ & -0.167$^{***}$ & -0.082$^{***}$ & -0.092$^{***}$\\   
                                & (0.008)        & (0.009)        & (0.011)        & (0.012)\\   
       \\
      Observations              & 11,760         & 11,933         & 11,901         & 11,382\\  
      R$^2$                     & 0.94506        & 0.93146        & 0.89717        & 0.89362\\  
      Adjusted R$^2$            & 0.94263        & 0.92848        & 0.89268        & 0.88875\\  
      \bottomrule
   \end{tabular}
   
   \par \raggedright 
   \caption{\label{tab:estadj_vol} Estimation for awarded bids: Adjusting by volume. This table shows estimates for awarded bids when prices are adjusted by volume following \eqref{eq:reg_sub_bid_rob}. Each model is defined by the type of method used for outlier detection according to Appendix~\ref{app:data}. Standardized product and region fixed effect are included in all models. We compute \textit{robust} standard errors.}
\end{table}




\subsection{Regressions for posted prices}


\begin{table}[H]
   \centering
   \begin{tabular}{lccc}
      \toprule
                                & Type 1         & Type 2         & Type 3 \\   
                                & (1)            & (2)            & (3)\\  
      \midrule 
      New                       & -0.029$^{***}$ & -0.033$^{***}$ & -0.025$^{***}$\\   
                                & (0.0007)       & (0.0008)       & (0.0008)\\   
      New $\times$ Comp  & -0.076$^{***}$ & -0.075$^{***}$ & -0.092$^{***}$\\   
                                & (0.0009)       & (0.0009)       & (0.001)\\   
       \\
      Observations              & 1,012,598      & 1,018,778      & 973,195\\  
      R$^2$                     & 0.97025        & 0.96865        & 0.96132\\  
      Adjusted R$^2$            & 0.97018        & 0.96857        & 0.96121\\  
      \bottomrule
   \end{tabular}
   
   \par \raggedright 
   \caption{Estimation for posted prices  under different outlier elimination rules. This table shows estimates for posted prices as in \eqref{eq:reg_op_posted}. Each Model is defined by the type of method used for outlier detection according to Appendix~\ref{app:data}. Standardized product, region, and week category-specific fixed effect are included in all models. We compute \textit{robust} standard errors.}
\end{table}



\subsection{Regressions for transaction prices}


\begin{table}[H]
   \centering
   \begin{tabular}{lccc}
      \toprule
                                & Type 1         & Type 2         & Type 3 \\   
                                & (1)            & (2)            & (3)\\  
      \midrule 
      New                       & -0.007$^{***}$ & -0.006$^{***}$ & 0.004$^{**}$\\   
                                & (0.002)        & (0.002)        & (0.002)\\   
      New $\times$ Comp  & -0.068$^{***}$ & -0.071$^{***}$ & -0.082$^{***}$\\   
                                & (0.002)        & (0.002)        & (0.002)\\   
       \\
      Observations              & 193,702        & 194,312        & 180,421\\  
      R$^2$                     & 0.97441        & 0.97328        & 0.97395\\  
      Adjusted R$^2$            & 0.97409        & 0.97294        & 0.97360\\  
      \bottomrule
   \end{tabular}
   
   \par \raggedright 
   \caption{Estimation for transaction prices. This table shows estimates for transaction prices as defined in Equation \eqref{eq:reg_op_posted}. Each model is defined by the type of method used for outlier detection according to Appendix~\ref{app:data}. Standardized product, region and week category-specific fixed effect are included in all models. We compute \textit{robust} standard errors.}
\end{table}



