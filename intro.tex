\begin{intro}

    In the last decade, the advancement of data science has significantly transformed various sectors, enabling more informed and efficient decision-making. In the public sector, the application of data analysis techniques offers unprecedented opportunities to improve public policy management, optimize resources, and respond more effectively to complex challenges. However, despite its potential, the implementation of data-based approaches in the public sector faces multiple obstacles, including limitations in technological infrastructure, cultural barriers, and the need for policies that ensure transparency and fairness in the use of information.
    
    The COVID-19 pandemic has highlighted the importance of agile and evidence-based decision-making in the public sector. Managing the health crisis has required implementing measures like lockdowns and mobility restrictions, whose effectiveness has varied considerably depending on socioeconomic factors. In Santiago, Chile, the capital of a country with significant socioeconomic inequalities, these measures have shown mixed results. Using detailed geolocated mobile phone data, mobility patterns during lockdowns were analyzed, revealing that high-income areas reduced their mobility by 50–90\% mainly through voluntary confinement behaviors, compared to 20–50\% in low-income communities. These findings were developed collaboratively with other researchers \cite{carranza2022social}, establishing a direct relationship between increased mobility and higher infection rates, where a 10\% increase in mobility is associated with a 5\% increase in infection rates. This work not only highlighted the importance of socioeconomic disparities in compliance with health measures but also provided insights that informed the implementation of more targeted public health policies.

    Furthermore, improving public procurement processes has been key to reducing costs and making government spending more efficient. In Chile, the framework agreements managed by ChileCompra represent a considerable portion of public spending, specifically 23\% of procurement expenses during 2018-19. However, inefficiencies were identified related to low competition in the auction processes of some framework agreements, which could potentially increase government costs. Framework agreements (FAs) are procurement mechanisms that offer an intermediate approach between centralized and decentralized purchasing, allowing a central agency to select a range of products and suppliers through auctions. Despite their advantages in terms of cost control and leveraging economies of scale, the lack of competition in auctions can lead to higher prices and increased costs for the government.
    
    To address these challenges, collaboration with ChileCompra was undertaken to redesign the FAs, implementing data-based reforms that included the standardization of product catalogs through natural language processing (NLP) algorithms and promoting greater competition in supplier auctions. The redesign process and its outcomes were developed in partnership with other researchers \cite{olivares2024saving}, demonstrating that the pilot redesign of the Food framework agreement achieved an 8\% reduction in transaction prices. This motivated ChileCompra to implement the developed tools more broadly, estimating a savings of approximately \$74 million in 2022. These findings provided significant guidelines for the practical implementation of FAs and contributed to the literature on applied market design.

    This research not only contributes to the existing state of the art but also has practical implications for those responsible for formulating public policies in Chile and in other similar contexts. By showing how data-based approaches can improve the efficiency and effectiveness of government interventions, this thesis offers a replicable model to face future challenges in the public sector. In the field of public health, the findings on mobility and socioeconomic disparities can inform more equitable and effective strategies to manage future health crises. In the area of public procurement, the proposed reforms are serving as a guide for other institutions seeking to optimize their processes and reduce costs through data-based approaches. Additionally, thanks to this work, modifications were made to the State procurement and contracting law, approved by the Senate of Chile in June 2023, which will allow for a simpler and more direct implementation of the tools developed in this work.
    
    In summary, this thesis explores how data science can transform decision-making in the public sector through two case studies focused on managing the COVID-19 pandemic and optimizing public procurement processes in Chile. The findings highlight the value of data-based approaches to address complex challenges, improving public policy outcomes and generating significant savings for the state. Additionally, the organization of the thesis is detailed: Chapter 1 details the study on optimizing public procurement, focusing on the framework agreements managed by ChileCompra and the reforms implemented to improve competition in auctions and reduce transaction costs. On the other hand, Chapter 2 explores in depth the management of the COVID-19 pandemic, analyzing mobility patterns and their impact on infection rates, as well as socioeconomic disparities in compliance with health measures. Each chapter presents the methodology employed, the results obtained, and the practical implications of the findings, providing a comprehensive view of how data science can improve decision-making in the public sector.
    
    
    
\end{intro}
