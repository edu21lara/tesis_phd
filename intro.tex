\begin{intro}

% En las últimas décadas, el uso de la ciencia de datos ha transformado la forma en que las organizaciones toman decisiones. Las compañías más exitosas destacan por su capacidad para tomar decisiones informadas y motivadas por el análisis de datos, lo que les permite anticiparse a cambios, optimizar procesos y mantenerse competitivas en el mercado. Del mismo modo, los gobiernos e instituciones públicas han comenzado a aprovechar estas herramientas, adoptando un enfoque de toma de decisiones basado en datos. Este cambio permite a las instituciones enfrentar problemas complejos con mayor precisión y efectividad, desde la gestión de crisis sanitarias hasta la mejora de procesos en la administración pública.

% En este contexto, la presente tesis se centra en dos áreas aplicadas de vital importancia para el desarrollo social y económico de los países: la gestión de la pandemia de COVID-19 y el rediseño de mecanismos de contratación pública. Estos dos casos muestran cómo el uso de datos no solo permite mejorar la toma de decisiones en situaciones críticas, sino también aumentar la eficiencia, transparencia y equidad en los procesos del Estado.

% La primera investigación se centra en un análisis detallado de la movilidad durante la pandemia en la ciudad de Santiago, lo que permitió identificar disparidades en la respuesta a las políticas de confinamiento y distanciamiento social. (Antes: En primer lugar, el análisis detallado de la movilidad durante la pandemia en la ciudad de Santiago permitió identificar disparidades en la respuesta a las políticas de confinamiento y distanciamiento social.) Se observó que la adherencia a las medidas sanitarias variaba significativamente según el nivel socioeconómico de las comunas. Este análisis granular fue clave para entender por qué ciertos sectores experimentaron mayores tasas de infección, lo que, a su vez, proporcionó información crucial para diseñar políticas de salud pública más focalizadas y efectivas.

% En segundo lugar, la tesis aborda la necesidad de optimizar los procesos de contratación pública, con especial atención al ``Convenio Marco'' en Chile. Utilizando técnicas avanzadas de ciencia de datos, se identificaron ineficiencias y cuellos de botella en el sistema, lo que permitió proponer reformas que mejoran la transparencia y la eficiencia del gasto público. Esta área de estudio es de especial relevancia, ya que un uso más eficiente de los recursos públicos puede traducirse en mejores servicios para la población.

% Ambos casos reflejan cómo la ciencia de datos puede ser una herramienta poderosa para resolver problemas estructurales en el sector público, mejorando la eficiencia en tiempos de crisis y en el funcionamiento cotidiano del Estado. Esta investigación destaca la importancia de desarrollar capacidades analíticas en las instituciones para enfrentar desafíos futuros de manera más informada y eficiente.

% (EN - Ingles)

In recent decades, the use of data science has transformed the way organizations make decisions. The most successful companies excel in their ability to make informed, data-driven decisions, allowing them to anticipate changes, optimize processes, and remain competitive in the market. Similarly, governments and public institutions have started using these tools, adopting a data-driven approach to decision-making. This shift enables institutions to tackle complex problems with greater precision and effectiveness, ranging from managing public health crises to improving administrative processes.

In this context, this thesis focuses on two applied areas of vital importance for the social and economic development of countries: the management of the COVID-19 pandemic and the redesign of public procurement mechanisms. These two cases demonstrate how the use of data not only improves decision-making in critical situations but also enhances efficiency, transparency, and fairness in government processes.

The first study focuses on a detailed analysis of mobility during the pandemic in the city of Santiago, which allowed for the identification of disparities in the response to lockdown and social distancing policie. It was observed that adherence to public health measures varied significantly based on the socioeconomic status of different neighborhoods. This granular analysis was key to understanding why certain areas experienced higher infection rates, which in turn provided crucial insights for designing more targeted and effective public health policies.

The second study addresses the need to optimize public procurement processes, with a special focus on the “Convenio Marco” in Chile. Using advanced data science techniques, inefficiencies and bottlenecks in the system were identified, leading to proposals for reforms that improve transparency and the efficiency of public spending. This area of study is particularly relevant, as more efficient use of public resources can translate into better services for the population.

Both cases reflect how data science can be a powerful tool to solve structural problems in the public sector, improving efficiency in times of crisis and in the everyday functioning of the state. This research highlights the importance of developing analytical capacities within institutions to better address future challenges in a more informed and efficient manner.

\end{intro}