\begin{intro}

    % % ------------------------------------------------------------
    % % -- Nueva Intro (Lun 21 octubre) ----------------------------
    % En la última década, el avance de la ciencia de datos ha transformado significativamente diversos sectores, permitiendo una toma de decisiones más informada y eficiente. En el sector público, la aplicación de técnicas de análisis de datos ofrece oportunidades sin precedentes para mejorar la gestión de políticas públicas, optimizar recursos y responder de manera más efectiva a desafíos complejos. Sin embargo, a pesar de su potencial, la implementación de enfoques basados en datos en el sector público enfrenta múltiples obstáculos, incluyendo limitaciones en la infraestructura tecnológica, barreras culturales y la necesidad de políticas que aseguren la transparencia y equidad en el uso de la información.
    
    % La pandemia de COVID-19 ha subrayado la importancia de la toma de decisiones ágiles y basadas en evidencia en el sector público. La gestión de la crisis sanitaria ha requerido la implementación de medidas como confinamientos y restricciones de movilidad, cuya efectividad ha variado considerablemente según factores socioeconómicos. En Santiago, Chile, la capital de un país con marcadas desigualdades socioeconómicas, estas medidas han mostrado resultados heterogéneos. Utilizando datos granulares de teléfonos móviles geolocalizados, se analizaron los patrones de movilidad durante los confinamientos, revelando que las áreas de altos ingresos redujeron su movilidad en un 50–90\% principalmente a través de comportamientos voluntarios de confinamiento, en comparación con un 20–50\% en comunidades de bajos ingresos. Además, se estableció una relación directa entre el aumento de la movilidad y el incremento de las tasas de infección, donde un aumento del 10\% en movilidad se asocia con un incremento del 5\% en las tasas de infección. Estos hallazgos permitieron la implementación de políticas de salud pública más focalizadas y destacaron la necesidad de medidas de apoyo en áreas de bajos ingresos para mejorar la adherencia a las medidas restrictivas.
    
    % Además, mejorar los procesos de contratación pública ha sido clave para reducir costos y hacer más eficiente el gasto del gobierno. En Chile, los convenios marco gestionados por ChileCompra representan una porción considerable del gasto público, específicamente el 23\% de los gastos de contratación durante 2018-19. Sin embargo, se identificaron ineficiencias relacionadas con la baja competencia en los procesos de subasta de algunos convenios marco, lo que potencialmente incrementa los costos gubernamentales. Los convenios marco (FAs) son mecanismos de contratación que ofrecen un enfoque intermedio entre la centralización y descentralización de compras, permitiendo que una agencia central seleccione una gama de productos y proveedores a través de subastas. A pesar de sus ventajas en términos de control de gastos y aprovechamiento de economías de escala, la falta de competencia en las subastas puede llevar a precios más altos y mayores costos para el gobierno.
    
    % Para abordar estos desafíos, se colaboró con ChileCompra en el rediseño de los FAs, implementando reformas basadas en datos que incluían la estandarización de catálogos de productos mediante algoritmos de procesamiento de lenguaje natural (NLP) y la promoción de una mayor competencia en las subastas de proveedores. Como resultado del piloto del rediseño del convenio marco de Alimentos, se observó una reducción del 8\% en los precios de transacción, lo que motivó a ChileCompra a una implementación más amplia de las herramientas desarrolladas durante el piloto, lo que se estima generó un ahorro de aproximadamente \$74 millones en 2022. Estos resultados demostraron el valor de restringir la competencia en la etapa de subasta del FA, proporcionando directrices importantes para la implementación de FAs en la práctica y contribuyendo a la literatura sobre diseño de mercados aplicados.
    
    % Esta investigación no solo aporta al estado del arte existente, sino que también tiene implicaciones prácticas para los responsables de la formulación de políticas públicas en Chile y en otros contextos similares. Al demostrar cómo los enfoques basados en datos pueden mejorar la eficiencia y efectividad de las intervenciones gubernamentales, esta tesis ofrece un modelo replicable para enfrentar desafíos futuros en el sector público. En el ámbito de la salud pública, los hallazgos sobre la movilidad y las disparidades socioeconómicas pueden informar estrategias más equitativas y efectivas para manejar futuras crisis sanitarias. En el área de contratación pública, las reformas propuestas están sirviendo como guía para otras instituciones que buscan optimizar sus procesos y reducir costos mediante enfoques basados en datos. Además, gracias a este trabajo, se realizaron modificaciones en la ley de compras y contrataciones del Estado, aprobada por el Senado de Chile en junio de 2023, lo que permitirá una implementación más sencilla y directa de las herramientas desarrolladas en este trabajo.
    
    % En resumen, esta tesis explora cómo la ciencia de datos puede transformar la toma de decisiones en el sector público a través de dos estudios de caso focalizados en la gestión de la pandemia de COVID-19 y la optimización de los procesos de contratación pública en Chile. Los hallazgos destacan el valor de los enfoques basados en datos para abordar desafíos complejos, mejorando los resultados de las políticas públicas y generando ahorros significativos para el Estado. Además, se detalla la organización de la tesis: el Capítulo 1 explora en profundidad la gestión de la pandemia de COVID-19, analizando los patrones de movilidad y su impacto en las tasas de infección, así como las disparidades socioeconómicas en el cumplimiento de las medidas sanitarias. Por otro lado, el Capítulo 2 detalla el estudio sobre la optimización de las compras públicas, enfocándose en los convenios marco gestionados por ChileCompra y las reformas implementadas para mejorar la competencia en las subastas y reducir los costos de transacción. Cada capítulo presenta la metodología empleada, los resultados obtenidos y las implicaciones prácticas de los hallazgos, proporcionando una visión integral de cómo la ciencia de datos puede mejorar la toma de decisiones en el sector público.
    
    % % (Ingles - Lun 21 de Oct)
    In the last decade, the advancement of data science has significantly transformed various sectors, enabling more informed and efficient decision-making. In the public sector, the application of data analysis techniques offers unprecedented opportunities to improve public policy management, optimize resources, and respond more effectively to complex challenges. However, despite its potential, the implementation of data-based approaches in the public sector faces multiple obstacles, including limitations in technological infrastructure, cultural barriers, and the need for policies that ensure transparency and fairness in the use of information.

    The COVID-19 pandemic has highlighted the importance of agile and evidence-based decision-making in the public sector. Managing the health crisis has required implementing measures like lockdowns and mobility restrictions, whose effectiveness has varied considerably depending on socioeconomic factors. In Santiago, Chile, the capital of a country with significant socioeconomic inequalities, these measures have shown mixed results. Using detailed geolocated mobile phone data, mobility patterns during lockdowns were analyzed, revealing that high-income areas reduced their mobility by 50–90\% mainly through voluntary confinement behaviors, compared to 20–50\% in low-income communities. Additionally, a direct relationship was found between increased mobility and higher infection rates, where a 10\% increase in mobility is associated with a 5\% increase in infection rates. These findings allowed for the implementation of more targeted public health policies and highlighted the need for support measures in low-income areas to improve adherence to restrictive measures.
    
    Furthermore, improving public procurement processes has been key to reducing costs and making government spending more efficient. In Chile, the framework agreements managed by ChileCompra represent a considerable portion of public spending, specifically 23\% of procurement expenses during 2018-19. However, inefficiencies were identified related to low competition in the auction processes of some framework agreements, which could potentially increase government costs. Framework agreements (FAs) are procurement mechanisms that offer an intermediate approach between centralized and decentralized purchasing, allowing a central agency to select a range of products and suppliers through auctions. Despite their advantages in terms of cost control and leveraging economies of scale, the lack of competition in auctions can lead to higher prices and increased costs for the government.
    
    To address these challenges, collaboration with ChileCompra was undertaken to redesign the FAs, implementing data-based reforms that included the standardization of product catalogs through natural language processing (NLP) algorithms and promoting greater competition in supplier auctions. As a result of the pilot redesign of the Food framework agreement, an 8\% reduction in transaction prices was observed, which motivated ChileCompra to implement the tools developed during the pilot more broadly, estimating a savings of approximately \$74 million in 2022. These results demonstrated the value of restricting competition in the FA auction stage, providing important guidelines for the practical implementation of FAs and contributing to the literature on applied market design.
    
    This research not only contributes to the existing state of the art but also has practical implications for those responsible for formulating public policies in Chile and in other similar contexts. By showing how data-based approaches can improve the efficiency and effectiveness of government interventions, this thesis offers a replicable model to face future challenges in the public sector. In the field of public health, the findings on mobility and socioeconomic disparities can inform more equitable and effective strategies to manage future health crises. In the area of public procurement, the proposed reforms are serving as a guide for other institutions seeking to optimize their processes and reduce costs through data-based approaches. Additionally, thanks to this work, modifications were made to the State procurement and contracting law, approved by the Senate of Chile in June 2023, which will allow for a simpler and more direct implementation of the tools developed in this work.
    
    In summary, this thesis explores how data science can transform decision-making in the public sector through two case studies focused on managing the COVID-19 pandemic and optimizing public procurement processes in Chile. The findings highlight the value of data-based approaches to address complex challenges, improving public policy outcomes and generating significant savings for the state. Additionally, the organization of the thesis is detailed: Chapter 1 details the study on optimizing public procurement, focusing on the framework agreements managed by ChileCompra and the reforms implemented to improve competition in auctions and reduce transaction costs. On the other hand, Chapter 2 explores in depth the management of the COVID-19 pandemic, analyzing mobility patterns and their impact on infection rates, as well as socioeconomic disparities in compliance with health measures. Each chapter presents the methodology employed, the results obtained, and the practical implications of the findings, providing a comprehensive view of how data science can improve decision-making in the public sector.
    
    
    
\end{intro}
